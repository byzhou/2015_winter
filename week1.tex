\documentclass[]{article}
\usepackage{graphicx}

\parindent=0pt
\usepackage[margin=0.5in]{geometry}
\usepackage{url}

\begin{document}
\pagestyle{empty}
{\large\textbf{Research Notes}}
\begin{itemize}
    \item[*] Created on Mon 04 Jan 2016 11:23:25 AM EST
    \item[*] Modified on \today
    \item[*] Author info: Boyou Zhou\\
             8 St Mary's St, PHO 340, Boston, MA 02215\\
             Email: bobzhou@bu.edu, Phone: 617-678-8480
\end{itemize}


\rule[-0.1cm]{7.5in}{0.01cm}\\
\\
\noindent \textbf{Mon 04 Jan 2016 11:14:43 AM EST}
\textit{Beginner's tutorial of AXI slave logic design}

This documents gives a brief instruction of how to design an AXI slave logic on
Zedboard. Zedboard provides two ARM cores and programmable logic. From here, ps
stands for GPP, ARM in this case, pl for programmable logic.\ 

Ps and Pl are connected with AXI ports. In this design, our logic talks to the
ps with AXI ports.\ ps looks at the logic as memory blocks. More strictly, the ps
visit the logic registers as memory units. On the ps side, I used a full linux,
Linaro Linux as the os for ps. The programs are run on the ps interfacing the
memory units through AXI ports, $/dev/mem$

Here are the tutorials to the design.
\indent		\begin{itemize}
            \item \textit{Start Linux} \url{http://fpga.org/2013/05/24/yet-another-guide-to-running-linaro-ubuntu-desktop-on-xilinx-zynq-on-the-zedboard/}
			\item \textit{AXI Slave Logic} \url{http://www.fpgadeveloper.com/2014/08/creating-a-custom-ip-block-in-vivado.html}
			\item \textit{Visit Memory Block} \url{http://fpga.org/2013/05/28/how-to-design-and-access-a-memory-mapped-device-part-one/}
        \end{itemize}

The device tree design, we do not include adv7511.dtsi in zynq-zed-adv7511.dtsi
\end{document}


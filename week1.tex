\documentclass[]{article}
\usepackage{graphicx}
\usepackage{xcolor}

\parindent=0pt
\usepackage[margin=0.5in]{geometry}
\usepackage{url}
\usepackage{hyperref}

\begin{document}
\pagestyle{empty}
{\large\textbf{Research Notes}}
\begin{itemize}
    \item[*] Created on Mon 04 Jan 2016 11:23:25 AM EST
    \item[*] Modified on \today
    \item[*] Author info: Boyou Zhou\\
             8 St Mary's St, PHO 340, Boston, MA 02215\\
             Email: bobzhou@bu.edu, Phone: 617-678-8480\\
\end{itemize}

\rule[-0.1cm]{7.5in}{0.01cm}\\
\noindent \textbf{Mon 04 Jan 2016 11:14:43 AM EST}
\textit{Beginner's tutorial of AXI slave logic design}

This documents gives a brief instruction of how to design an AXI slave logic on
Zedboard. Zedboard provides two ARM cores and programmable logic. From here, ps
stands for GPP, ARM in this case, pl for programmable logic.\ 

Ps and Pl are connected with AXI ports. In this design, our logic talks to the
ps with AXI ports.\ ps looks at the logic as memory blocks. More strictly, the ps
visit the logic registers as memory units. On the ps side, I used a full linux,
Linaro Linux as the os for ps. The programs are run on the ps interfacing the
memory units through AXI ports, $/dev/mem$

Here are the tutorials to the design.
\indent		\begin{itemize}
			\item \textit{Start Linux}
			\url{http://fpga.org/2013/05/24/yet-another-guide-to-running-linaro-ubuntu-desktop-on-xilinx-zynq-on-the-zedboard/}
			\item \textit{AXI Slave Logic}
			\url{http://www.fpgadeveloper.com/2014/08/creating-a-custom-ip-block-in-vivado.html}
			\item \textit{Visit Memory Block}
			\url{http://fpga.org/2013/05/28/how-to-design-and-access-a-memory-mapped-device-part-one/}
        \end{itemize}

The device tree design, we do not include adv7511.dtsi in zynq-zed-adv7511.dtsi

\noindent \textbf{Thu 07 Jan 2016 12:38:46 PM EST}
\textit{Papers related to security}
\begin{itemize}
	\item side channel detection \cite{longo2015soc} 
	\item encoding methods \cite{chakraborti2015trivia} 
	\item radio security breach \cite{genkin2015stealing}
	\item PUF \cite{aysu2015end} \cite{maes2015secure} \cite{herder2014physical} \cite{devadas2010secure}
	\item SAT \cite{saha2015improved}
	\item TRNG \cite{haddad2015physical}\cite{suh2007physical}\cite{herder2014trapdoor}
	\item new tech \cite{suh2003efficient}
	
\rule[-0.1cm]{7.5in}{0.01cm}\\
\\
	\item coprocessor \cite{roy2015lightweight} 
	\item break RSA on Intel chip \cite{bhattacharya2015watches}
	\item accelerating homomorphic encryption \cite{doroz2015accelerating}
	\item memory verification and encryption, verification ensures the
adversary changes in the machine states. Encryption protects the off-chip
memory\cite{suh2003efficient}

\end{itemize}

some explanation
\begin{itemize}
	\item \cite{herder2014trapdoor}TRNG: safely extract the keys from biometric source
\end{itemize}

\rule[-0.1cm]{7.5in}{0.01cm}\\
\noindent \textbf{Mon 01 Feb 2016 09:55:39 AM EST}
\textit{Architecture people working on security}

\begin{itemize}
	\item Srini Devdas (MIT) \url{https://scholar.google.com/citations?user=-yrzguMAAAAJ&hl=en}
	\item Edward Suh (Cornell) \url{https://scholar.google.com/citations?user=neO3vFYAAAAJ&hl=en&oi=ao}~\cite{chen2015execution}
	\item Mohit Tiwari (U T Austin)
	\item Simha Sethumadhavan (Columbia)
	\item Tim Sherwood (UCSB)
	\item Dawn Song (U C Berkeley)
\end{itemize}

In the homomorphic computation, somewhat homomorphic function evaluation needs to be reviewed.\textbf{SHF}
\begin{itemize}
	\item \cite{chen2015execution} gives the 
\end{itemize}

\rule[-0.1cm]{7.5in}{0.01cm}\\
\noindent \textbf{Mon 22 Feb 2016 01:59:29 PM EST}
\textit{papers on architecture security}
\begin{itemize}
	\item \cite{wang2014timing} RSA attacks: each RSA decoding will result in
memory access. Thus the number of access in memory is the hamming distance of
RSA keys.
	\item \cite{ismail2015improving} 
	\item \cite{ancajas2014fort} thread model: HTs attacks one of the cores and
propagate information to other cores.  
        \begin{itemize}
            \item data scrambling
            \item dynamic packet certificate
            \item node obfuscation
        \end{itemize}
    \item \cite{liu2015ghostrider} 
        \begin{itemize}
            \item split the memory into three types, normal memory, encrypted memory and oblivious memory
            \item use software-directed scratchpad instead of implicit cache access
            \item deterministic processor pipeline against timing attacks
        \end{itemize}
\end{itemize}

\rule[-0.1cm]{7.5in}{0.01cm}\\
\noindent \textbf{Mon 22 Feb 2016 01:59:29 PM EST}
\textit{papers on architecture security}
\begin{itemize}
    \item \cite{diguet2007noc} Threat models:
        \begin{itemize}
            \item a write access in the secure area to modify the system behavior
            \item covert attacks
                \begin{itemize}
                    \item extraction of information
                        \begin{itemize}
                            \item RSA fault injection\cite{pellegrini2010fault} 
                            \item AES attack\cite{moradi2006generalized}
                            \item DPA \cite{kocher2011introduction}
                        \end{itemize}
                    \item communication between different applications \cite{wang2012efficient}
                \end{itemize}
            \item denial of service:
                \begin{itemize}
                    \item replay: wastes of bandwidth
                    \item incorrect path: introduce erroneous paths
                    \item deadlock: the use of packets with paths that intentionally create deadlocks
                    \item livelock: introduce packets that can never reach the end so that they stay turning in the network
                \end{itemize}
        \end{itemize}
\end{itemize}

\rule[-0.1cm]{7.5in}{0.01cm}\\
\noindent \textbf{Tue 08 Mar 2016 02:28:51 PM EST}
\begin{itemize}
    \item SNI - \textit{Secure Network Interface} 
        \begin{itemize} 
            \item DoS
            \item Unauthorized Read Access(Information Extraction) and unauthorized write access (Hijacking){\color{red}more literature review}
        \end{itemize}        
    \item bus-based control~\cite{diguet2007noc}
        \begin{itemize}
            \item overflow checking
            \item path based identification
            \item local access checking
            \item statistics
        \end{itemize}
    \item secure memory access \cite{goossens2008hardwired} extending \cite{fiorin2007data}
        \begin{itemize}
            \item design of Data Protection Units (DPUs) managed by Network Security Manager(NSM)
            \item memory security \cite{gebotys2003security}
            \item anti-side channel \cite{evaluation}
        \end{itemize} 
\end{itemize}

\begin{itemize}
	\item \cite{kocher2011introduction} introduction to differential power analysis
	\item \cite{pellegrini2010fault} RSA attack, retrieve the private key with fault based  
\end{itemize}

\bibliography{week1}{}
\bibliographystyle{plain}
\end{document}

